\documentclass[10pt, a4paper, twocolumn]{article}

%%%%%%%%%%%%%%%%%%%%%%%%%%%%%%%%%%%%%%%%%
% Wenneker Article
% Structure Specification File
% Version 1.0 (28/2/17)
%
% This file originates from:
% http://www.LaTeXTemplates.com
%
% Authors:
% Frits Wenneker
% Vel (vel@LaTeXTemplates.com)
%
% License:
% CC BY-NC-SA 3.0 (http://creativecommons.org/licenses/by-nc-sa/3.0/)
%
%%%%%%%%%%%%%%%%%%%%%%%%%%%%%%%%%%%%%%%%%

%----------------------------------------------------------------------------------------
%	PACKAGES AND OTHER DOCUMENT CONFIGURATIONS
%----------------------------------------------------------------------------------------

\usepackage[english]{babel} % English language hyphenation

\usepackage{microtype} % Better typography

\usepackage{amsmath,amsfonts,amsthm} % Math packages for equations

\usepackage[svgnames]{xcolor} % Enabling colors by their 'svgnames'

\usepackage[hang, small, labelfont=bf, up, textfont=it]{caption} % Custom captions under/above tables and figures

\usepackage{booktabs} % Horizontal rules in tables

\usepackage{lastpage} % Used to determine the number of pages in the document (for "Page X of Total")

\usepackage{graphicx} % Required for adding images

\usepackage{enumitem} % Required for customising lists
\setlist{noitemsep} % Remove spacing between bullet/numbered list elements

\usepackage{sectsty} % Enables custom section titles
\allsectionsfont{\usefont{OT1}{phv}{b}{n}} % Change the font of all section commands (Helvetica)

\definecolor{arsenic}{rgb}{0.23, 0.27, 0.29}

%----------------------------------------------------------------------------------------
%	MARGINS AND SPACING
%----------------------------------------------------------------------------------------

\usepackage{geometry} % Required for adjusting page dimensions

\geometry{
top=1cm, % Top margin
bottom=1.5cm, % Bottom margin
left=2cm, % Left margin
right=2cm, % Right margin
includehead, % Include space for a header
includefoot, % Include space for a footer
%showframe, % Uncomment to show how the type block is set on the page
}

\setlength{\columnsep}{7mm} % Column separation width

%----------------------------------------------------------------------------------------
%	FONTS
%----------------------------------------------------------------------------------------

\usepackage[T1]{fontenc} % Output font encoding for international characters
\usepackage[utf8]{inputenc} % Required for inputting international characters

%\usepackage{XCharter} % Use the XCharter font

%----------------------------------------------------------------------------------------
%	HEADERS AND FOOTERS
%----------------------------------------------------------------------------------------

\usepackage{fancyhdr} % Needed to define custom headers/footers
\pagestyle{fancy} % Enables the custom headers/footers

\renewcommand{\headrulewidth}{0.0pt} % No header rule
\renewcommand{\footrulewidth}{0.4pt} % Thin footer rule

\renewcommand{\sectionmark}[1]{\markboth{#1}{}} % Removes the section number from the header when \leftmark is used

%\nouppercase\leftmark % Add this to one of the lines below if you want a section title in the header/footer

% Headers
\lhead{} % Left header
\chead{\textit{\thetitle}} % Center header - currently printing the article title
\rhead{} % Right header

% Footers
\lfoot{} % Left footer
\cfoot{} % Center footer
\rfoot{\footnotesize Page \thepage\ of \pageref{LastPage}} % Right footer, "Page 1 of 2"

\fancypagestyle{firstpage}{ % Page style for the first page with the title
\fancyhf{}
\renewcommand{\footrulewidth}{0pt} % Suppress footer rule
}

%----------------------------------------------------------------------------------------
%	TITLE SECTION
%----------------------------------------------------------------------------------------

\newcommand{\authorstyle}[1]{{\large\usefont{OT1}{phv}{b}{n}\color{arsenic}#1}} % Authors style (Helvetica)

\newcommand{\institution}[1]{{\footnotesize\usefont{OT1}{phv}{m}{sl}\color{Black}#1}} % Institutions style (Helvetica)

\usepackage{titling} % Allows custom title configuration

\newcommand{\HorRule}{\color{DarkGoldenrod}\rule{\linewidth}{1pt}} % Defines the gold horizontal rule around the title

\pretitle{
\vspace{-30pt} % Move the entire title section up
\HorRule\vspace{10pt} % Horizontal rule before the title
\fontsize{32}{36}\usefont{OT1}{phv}{b}{n}\selectfont % Helvetica
\color{arsenic} % Text colour for the title and author(s)
%	\color{Dark cerulean} % Text colour for the title and author(s)
%	\color{Arsenic} % Text colour for the title and author(s)
}

\posttitle{\par\vskip 15pt} % Whitespace under the title

\preauthor{} % Anything that will appear before \author is printed

\postauthor{ % Anything that will appear after \author is printed
\vspace{10pt} % Space before the rule
\par\HorRule % Horizontal rule after the title
\vspace{20pt} % Space after the title section
}

%----------------------------------------------------------------------------------------
%	ABSTRACT
%----------------------------------------------------------------------------------------

\usepackage{lettrine} % Package to accentuate the first letter of the text (lettrine)
\usepackage{fix-cm}    % Fixes the height of the lettrine

\newcommand{\initial}[1]{ % Defines the command and style for the lettrine
\lettrine[lines=3,findent=4pt,nindent=0pt]{% Lettrine takes up 3 lines, the text to the right of it is indented 4pt and further indenting of lines 2+ is stopped
\color{DarkGoldenrod}% Lettrine colour
{#1}% The letter
}{}%
}

\usepackage{xstring} % Required for string manipulation

\newcommand{\lettrineabstract}[1]{
\StrLeft{#1}{1}[\firstletter] % Capture the first letter of the abstract for the lettrine
\initial{\firstletter}\textbf{\StrGobbleLeft{#1}{1}} % Print the abstract with the first letter as a lettrine and the rest in bold
}

%----------------------------------------------------------------------------------------
%	BIBLIOGRAPHY
%----------------------------------------------------------------------------------------

\usepackage[backend=biber, natbib=true]{biblatex} % Use the bibtex backend with the authoryear citation style (which resembles APA)

\addbibresource{nephilia.bib} % The filename of the bibliography

\usepackage[autostyle=true]{csquotes} % Required to generate language-dependent quotes in the bibliography

\title{Nephilia: A new approach to recursive spectral clustering of bipartite graphs}

\author{
\authorstyle{Florian Schaefer}
}

\date{\today}

\begin{document}
    \maketitle
    \thispagestyle{firstpage}
    \lettrineabstract{
        NOTE: At this moment, this document is a huge mess and far from complete. It's only purpose right now is to retain
        some information about the use of spectral shifting. Please ignore it for now and come back once
        it has been put in a proper form.
    }

    \paragraph{Introduction}
    Data is the new oil they say.
    \\
    While only time will tell whether it will in fact exert a tantamount transformative force on all of society, there exist
    certain commonalities that can presumed to be satisfied already today.
    Just like oil, data is not manufactured, but instead extracted from deposits and wells where it came into being as an
    artifact of life itself.
    And just like oil, it arrives not only in vast quantities, but also in such an extremely crude, disorganized and impure
    state that it may very well be considered virtually worthless.
    That is, until it is refined and split up into its homogeneous constituent groups.
    It is not oil, but those substances that are at the basis of our everyday lives and that decide over the rise and fall
    of whole nation states.
    \\
    Hence, the initial pondering about the potential society-shaping impact that data might have one day is inextricably
    linked to the question of whether we can manage to reliably refine it on large scales.
    \\
    To put it more ambitiously: If we admit data the crucial importance we claim it deserves, we require tools that are
    not only able to isolate homogeneous and self-consistent subsets from arbitrary data, but that also allow modelling
    whole datasets in terms of relationships between those subsets.
    Moreover, we need generally intelligible criteria that manage to explain the data instead of forcing the user to
    accept whatever comes out of impenetrable black boxes.

    However, the current landscape of available and ready-to-use clustering algorithms only partially contributes to this goal.
    While some established methods (such as PCA, t-SNE or LDA) stand on very solid theoretical grounds, none of them
    actually provides a proper explanation or classification of the results that end users and decision makers might
    find satisfactory.

    Spectral clustering exhibits a few particular advantages that other clustering methods do not.
    For example, in terms of the outcome, spectral clustering is very well defined and completely deterministic.
    There is no need to guess the right amount of clusters as there are no magic initial values. But most
    importantly, spectral clustering works on relationships between things instead of metric spaces.
    This not only better reflects the nature of the input, but also allows for a fine-grained control of what the end
    result should look like.

    \section{Spectral shifting}

    \paragraph
    As pointed out earlier, we're only interested in the second-smallest eigenvalue of the normalized laplacian.
    While there exist many libraries (ARPack, Eigen, SLEPc, just to name a few) to perform this task, due to their
    generality, they lack the ability to exploit certain properties of the problem at hand. More precisely, both the extremal
    eigenvalues as well as the eigenvector associated with the smallest eigenvalue are known a priori.
    \\
    It shows that those properties can easily exploited to transform the matrix at hand such that the new matrix
    associates the desired eigenvector with its largest eigenvalue. This is performed using a well-known technique called spectral shifting.

    % TODO: Check matrix conditions for both theorems (symmetric? positive-definite? .... v != 0, ...)
    \subsection{Spectral transformations}

    % https://math.stackexchange.com/questions/2214641/shifting-eigenvalues-of-a-matrix
    \newtheorem{eigenvalueReplacement}[]{Theorem: Eigenvalue Replacement}[section]
    \begin{eigenvalueReplacement}
        Let $A \in \mathbb{R}^{n \times n}$ be a real symmetric matrix, $n \geq i \in \mathbb{N}$ and $\lambda_i, v_i$
        the $i-th$ eigenvalue and its respective eigenvector of $A$. Then, for any $\mu \in \mathbb{R}$, the matrix
        \begin{align}
            \tilde{A} := A + \left(\mu - \lambda_i \right) v_i {v_i}^T
        \end{align}
        holds the exact same set of eigenvalues and eigenvectors as $A$, except for $\lambda_i$, which has been replaced by $\mu$.
        In particular, the eigenvector assigned with $\mu$ is the same as $v_i$.
        \begin{proof}[Proof]
            Since $A$ is real and symmetric, we know that there exists a decomposition $A=VDV^T$, where $V$ holds the orthonormal
            set of eigenvectors and $D$ being a diagonal matrix with the eigenvalues. With this in mind, we can write
            \begin{align}
                \begin{split}
                    \tilde{A} & = A + \left( \mu - \lambda_i \right) v_i {v_i}^T \\
                    & = A + \left( \mu - \lambda_i \right) v_i {e_i}^T V^T \\
                    & = A + \left( \mu - \lambda_i \right) \left( V e_i \right) {e_i}^T V^T \\
                    & = VDV^T + \left( \mu - \lambda_i \right) V \left( e_i {e_i}^T \right) V^T \\
                    & = V\left( D + \left( \mu - \lambda_i \right) e_i {e_i}^T \right) V^T
                \end{split}
            \end{align}
        \end{proof}
    \end{eigenvalueReplacement}

    \newtheorem{shifting}[]{Theorem: Spectral shifting (Mirroring special case)}[section]
    \begin{shifting}
        Let $A \in \mathbb{R}^{n \times n}$ be a real symmetric matrix and $\sigma \in \mathbb{R}$.
        Then, the matrix
        \begin{align}
            \hat{A} := \sigma I - A
        \end{align}
        Will be mirrored along $\sigma$. More precisely, for any eigenvalue $\lambda$ of $A$ and Eigenvector
        $v \in \mathbb{R}^n$, the following identity holds:
        \begin{align}
            \left( \sigma I -A \right) v = \left( \sigma - \lambda \right) v
        \end{align}
        Furthermore, the set of eigenvectors of $\hat{A}$ are precisely that of $A$.
        \begin{proof}[Proof]
            Let $V \in \mathbb{R}^{n \times n}$ be the orthonormal matrix consisting of all eigenvectors of $A$
            and $D \in \mathbb{R}^{n \times n}$ the diagonal matrix from its eigenvalues. Then:
            \begin{align}
                \begin{split}
                    V \left( \sigma I - D \right) V^T & = V \left( \sigma I V^T - D V ^T \right) \\
                    & = V \sigma I V^T - VDV^T \\
                    & = \sigma V V^T - A \\
                    & = \sigma I -A
                \end{split}
            \end{align}
        \end{proof}
    \end{shifting}

    With those tools at hand, we can now formulate the transformation that will assign the eigenvector of the
    second-smallest eigenvalue to the largest eigenvalue of the transformed matrix.
    \\
    Remember that there a few known properties of the normalized laplacian spectrum. In particular:
    \begin{itemize}
        \item We know that the smallest eigenvalue is $0$ with eigenvector $\frac{D^\frac{1}{2}}{\lVert D^\frac{1}{2} \rVert}$ % TODO: Wrong! There is no D here
        \item The largest eigenvalue is known to be $ \leq 2$.
    \end{itemize}

    \newtheorem{combined}[]{Corollary: Making the $v_2$ accessible }[section]
    \begin{shifting}
        Let $\mathcal{L} \in \mathbb{R}^{n \times n}$ be the normalized laplacian of any graph and $v_0$ the known
        smallest eigenvector belonging to the eigenvalue $0$. Then, the eigenvector associated with the largest
        eigenvalue of the following matrix is exactly that of the second-smalles eigenvalue of the regular
        normalized Laplacian
        \begin{align}
            \begin{split}
                \tilde{\mathcal{L}} & = \underbrace{2I -}_{mirror\ spectrum\ along\ 2} \underbrace{\left( \mathcal{L} + 2 v_0 {v_0}^T \right)}_{send\ \lambda_0 \rightarrow 2} \\
                & = 2 \left( I - v_0 {v_0}^T \right) - \mathcal{L}
            \end{split}
        \end{align}
    \end{shifting}

    \section{Bringing it all together}
    In \eqref{shiftedlaplacian}, we were presented with a simple analytic solution for any spectrally shifted normalized laplacian.
    Combining those results with those from (REF MISSING) to fill in the rest:
    \begin{equation}
        \begin{split}
            \hat{\mathcal{L}} & = 2 ( I - v_0 v_0^T ) - \mathcal{L}\\
            & = I + D^{-\frac{1}{2}} A D^{-\frac{1}{2}} - 2 v_0 v_0^T
        \end{split}
    \end{equation}
    New let's see how this operates on some vector $v \in \mathbb{R}^n$. This immediately yields some major optimizations
    \begin{equation}
        \begin{split}
            \hat{\mathcal{L}}v & = \left( I + D^{-\frac{1}{2}} A D^{-\frac{1}{2}} - 2 v_0 v_0^T\right) v\\
            & = v + D^{-\frac{1}{2}} A \underbrace{D^{-\frac{1}{2}} v}_{=:\hat{v}} - 2 v_0 \underbrace{v_0^T v}_{=:\frac{\mu}{2}}\\
            & = v + D^{-\frac{1}{2}} A \hat{v} - \mu v_0
        \end{split}
    \end{equation}
    Thus, the shifted laplacian operator can be realized as a standard matrix multiplication on the original matrix with both the
    argument and result vectors being slightly transformed. It it also worth noting that since $D$ is strictly diagonal, the multiplications with $D^{-\frac{1}{2}}$
    are actually linear operations in the number of rows. Also: The operator is sparsity-invariant with respect to $A$


    \section{The case against other eigensolvers}

    \printbibliography[title={Bibliography}]

\end{document}
