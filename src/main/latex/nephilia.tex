%----------------------------------------------------------------------------------------
%	PACKAGES AND OTHER DOCUMENT CONFIGURATIONS
%----------------------------------------------------------------------------------------

\documentclass[10pt, a4paper, twocolumn]{article} % 10pt font size (11 and 12 also possible), A4 paper (letterpaper for US letter) and two column layout (remove for one column)

%%%%%%%%%%%%%%%%%%%%%%%%%%%%%%%%%%%%%%%%%
% Wenneker Article
% Structure Specification File
% Version 1.0 (28/2/17)
%
% This file originates from:
% http://www.LaTeXTemplates.com
%
% Authors:
% Frits Wenneker
% Vel (vel@LaTeXTemplates.com)
%
% License:
% CC BY-NC-SA 3.0 (http://creativecommons.org/licenses/by-nc-sa/3.0/)
%
%%%%%%%%%%%%%%%%%%%%%%%%%%%%%%%%%%%%%%%%%

%----------------------------------------------------------------------------------------
%	PACKAGES AND OTHER DOCUMENT CONFIGURATIONS
%----------------------------------------------------------------------------------------

\usepackage[english]{babel} % English language hyphenation

\usepackage{microtype} % Better typography

\usepackage{amsmath,amsfonts,amsthm} % Math packages for equations

\usepackage[svgnames]{xcolor} % Enabling colors by their 'svgnames'

\usepackage[hang, small, labelfont=bf, up, textfont=it]{caption} % Custom captions under/above tables and figures

\usepackage{booktabs} % Horizontal rules in tables

\usepackage{lastpage} % Used to determine the number of pages in the document (for "Page X of Total")

\usepackage{graphicx} % Required for adding images

\usepackage{enumitem} % Required for customising lists
\setlist{noitemsep} % Remove spacing between bullet/numbered list elements

\usepackage{sectsty} % Enables custom section titles
\allsectionsfont{\usefont{OT1}{phv}{b}{n}} % Change the font of all section commands (Helvetica)

\definecolor{arsenic}{rgb}{0.23, 0.27, 0.29}

%----------------------------------------------------------------------------------------
%	MARGINS AND SPACING
%----------------------------------------------------------------------------------------

\usepackage{geometry} % Required for adjusting page dimensions

\geometry{
top=1cm, % Top margin
bottom=1.5cm, % Bottom margin
left=2cm, % Left margin
right=2cm, % Right margin
includehead, % Include space for a header
includefoot, % Include space for a footer
%showframe, % Uncomment to show how the type block is set on the page
}

\setlength{\columnsep}{7mm} % Column separation width

%----------------------------------------------------------------------------------------
%	FONTS
%----------------------------------------------------------------------------------------

\usepackage[T1]{fontenc} % Output font encoding for international characters
\usepackage[utf8]{inputenc} % Required for inputting international characters

%\usepackage{XCharter} % Use the XCharter font

%----------------------------------------------------------------------------------------
%	HEADERS AND FOOTERS
%----------------------------------------------------------------------------------------

\usepackage{fancyhdr} % Needed to define custom headers/footers
\pagestyle{fancy} % Enables the custom headers/footers

\renewcommand{\headrulewidth}{0.0pt} % No header rule
\renewcommand{\footrulewidth}{0.4pt} % Thin footer rule

\renewcommand{\sectionmark}[1]{\markboth{#1}{}} % Removes the section number from the header when \leftmark is used

%\nouppercase\leftmark % Add this to one of the lines below if you want a section title in the header/footer

% Headers
\lhead{} % Left header
\chead{\textit{\thetitle}} % Center header - currently printing the article title
\rhead{} % Right header

% Footers
\lfoot{} % Left footer
\cfoot{} % Center footer
\rfoot{\footnotesize Page \thepage\ of \pageref{LastPage}} % Right footer, "Page 1 of 2"

\fancypagestyle{firstpage}{ % Page style for the first page with the title
\fancyhf{}
\renewcommand{\footrulewidth}{0pt} % Suppress footer rule
}

%----------------------------------------------------------------------------------------
%	TITLE SECTION
%----------------------------------------------------------------------------------------

\newcommand{\authorstyle}[1]{{\large\usefont{OT1}{phv}{b}{n}\color{arsenic}#1}} % Authors style (Helvetica)

\newcommand{\institution}[1]{{\footnotesize\usefont{OT1}{phv}{m}{sl}\color{Black}#1}} % Institutions style (Helvetica)

\usepackage{titling} % Allows custom title configuration

\newcommand{\HorRule}{\color{DarkGoldenrod}\rule{\linewidth}{1pt}} % Defines the gold horizontal rule around the title

\pretitle{
\vspace{-30pt} % Move the entire title section up
\HorRule\vspace{10pt} % Horizontal rule before the title
\fontsize{32}{36}\usefont{OT1}{phv}{b}{n}\selectfont % Helvetica
\color{arsenic} % Text colour for the title and author(s)
%	\color{Dark cerulean} % Text colour for the title and author(s)
%	\color{Arsenic} % Text colour for the title and author(s)
}

\posttitle{\par\vskip 15pt} % Whitespace under the title

\preauthor{} % Anything that will appear before \author is printed

\postauthor{ % Anything that will appear after \author is printed
\vspace{10pt} % Space before the rule
\par\HorRule % Horizontal rule after the title
\vspace{20pt} % Space after the title section
}

%----------------------------------------------------------------------------------------
%	ABSTRACT
%----------------------------------------------------------------------------------------

\usepackage{lettrine} % Package to accentuate the first letter of the text (lettrine)
\usepackage{fix-cm}    % Fixes the height of the lettrine

\newcommand{\initial}[1]{ % Defines the command and style for the lettrine
\lettrine[lines=3,findent=4pt,nindent=0pt]{% Lettrine takes up 3 lines, the text to the right of it is indented 4pt and further indenting of lines 2+ is stopped
\color{DarkGoldenrod}% Lettrine colour
{#1}% The letter
}{}%
}

\usepackage{xstring} % Required for string manipulation

\newcommand{\lettrineabstract}[1]{
\StrLeft{#1}{1}[\firstletter] % Capture the first letter of the abstract for the lettrine
\initial{\firstletter}\textbf{\StrGobbleLeft{#1}{1}} % Print the abstract with the first letter as a lettrine and the rest in bold
}

%----------------------------------------------------------------------------------------
%	BIBLIOGRAPHY
%----------------------------------------------------------------------------------------

\usepackage[backend=biber, natbib=true]{biblatex} % Use the bibtex backend with the authoryear citation style (which resembles APA)

\addbibresource{nephilia.bib} % The filename of the bibliography

\usepackage[autostyle=true]{csquotes} % Required to generate language-dependent quotes in the bibliography
 % Specifies the document structure and loads requires packages

%----------------------------------------------------------------------------------------
%	ARTICLE INFORMATION
%----------------------------------------------------------------------------------------

\title{Nephilia: A new approach to recursive spectral clustering of bipartite graphs} % The article title

\author{
	\authorstyle{Florian Schaefer}
}

% Example of a one line author/institution relationship
%\author{\newauthor{John Marston} \newinstitution{Universidad Nacional Autónoma de México, Mexico City, Mexico}}

\date{\today} % Add a date here if you would like one to appear underneath the title block, use \today for the current date, leave empty for no date

%----------------------------------------------------------------------------------------

\begin{document}

\maketitle % Print the title

\thispagestyle{firstpage} % Apply the page style for the first page (no headers and footers)

%----------------------------------------------------------------------------------------
%	ABSTRACT
%----------------------------------------------------------------------------------------

\lettrineabstract{There goes the abstract. Please fill it with some live!}

%----------------------------------------------------------------------------------------
%	ARTICLE CONTENTS
%----------------------------------------------------------------------------------------

\paragraph{Introduction}
Spectral clustering exhibits a few particular beauties that other clustering methods do not.
For example, in terms of the outcome, spectral clustering is very well defined and completely deterministic.
There is no need to guess the right amount of clusters as there are no magic initial values. But most
importantly, spectral clustering works on relationships between things instead of metric spaces.
This not only better reflects the nature of the input, but also allows for a fine-grained control of what the
result should look like.

\section{Spectral shifting}

\paragraph
As pointed out earlier, we're only interested in the second-smallest eigenvalue of the normalized laplacian.
While there exist many libraries (ARPack, Eigen, SLEPc, just to name a few) to perform this task, due to their
generality, they lack the ability to exploit certain properties of the problem at hand. More precisely, both the extremal
eigenvalues as well as the eigenvector associated with the smallest eigenvalue are known a priori.
\\
It shows that those properties can easily exploited to transform the matrix at hand such that the new matrix
associates the desired eigenvector with its largest eigenvalue. This is performed using a well-known technique called spectral shifting.

% TODO: Check matrix conditions for both theorems (symmetric? positive-definite? .... v != 0, ...)
\subsection{Spectral transformations}

% https://math.stackexchange.com/questions/2214641/shifting-eigenvalues-of-a-matrix
\newtheorem{Brauer}[]{Theorem (Brauer)}[section]
\begin{Brauer}
	Let $A \in \mathbb{R}^{n \times n}$ a matrix with $Av=\lambda v$ for some eigenvalue $\lambda \in \mathbb{R}$ and $0 \neq v \in \mathbb{R}^n$.
	Furthermore, let $r \in \mathbb{R}^n$ with $r^{T}v=1$. Then, $\forall \mu \in \mathbb{R}:$ the eigenvalues of
	\begin{align}
		\hat{A} := A + (\lambda - \mu)vr^T
	\end{align}
	are exactly those of $A$, except for $\lambda$, which has been replaced by $\mu$. Furthermore, the eigenvector $v$ remains
	completely unchanged, i.e. $\hat{A}v = \mu v$
\end{Brauer}
This theorem will allow us later to swap specific eigenvalues.

\newtheorem{Brauer Param Note}[]{Note}[section]
\begin{Brauer Param Note}
Chosing $r:=v$ in the above theorem is explicitly allowed if $v$ is normalized.
\end{Brauer Param Note}

\newtheorem{Spectral Shifting}[]{Theorem (Spectral Shifting)}[section]
\begin{Spectral Shifting}
	Let $A \in \mathbb{R}^{n \times n}$ a symmetric, positive-definite matrix with $Av=\lambda v$ for some eigenvalue $\lambda \in \mathbb{R}$.
	Furthermore, let $\mu \in \mathbb{R}$. Then
	\begin{align}
		(A - \sigma I)v = (\lambda - \mu)v
	\end{align}
That is, $(\lambda - \mu)v$ is an eigenvalue of the Matrix $(A - \sigma I)v$. Just like before, this does not affect the
eigenvector $v$.
\end{Spectral Shifting}

\newtheorem{Reversal}[]{Corollary (Reverse Spectra)}[section]
\begin{Reversal}
	Obviously the reversed statement $(\sigma I - A)v = (\mu - \lambda)v$ also holds true.
	Let $A \in \mathbb{R}^{n \times n}$, $v$ like above. Let $\lambda_n$ the largest eigenvalue.
	It immediately follows that the transformation
	\begin{align}
		\hat{A} := \lambda_n I - A
	\end{align}
	will mirror the eigenvalue at $\lambda_n$. That is, the order of the spectrum is reversed without affecting any eigenvectors of $A$.
\end{Reversal}
With those tools at hand, we can now formulate the transformation that will assign the eigenvector of the
second-smallest eigenvalue to the largest eigenvalue of the transformed matrix.
\\
Remember that there a few known properties of the normalized laplacian spectrum. In particular:
\begin{itemize}
	\item We know that the smallest eigenvalue is $0$ with eigenvector $\frac{D^\frac{1}{2}}{\lVert D^\frac{1}{2} \rVert}$ % TODO: Wrong! There is no D here
	\item If the graph is bipartite, the largest eigenvalue is known to be $2$.
\end{itemize}

Now we can first reverse the spectrum along the largest eigenvalue $2$ and then (the order matters) send the
now-largest eigenvalue $2$ (that used to be $0$) to $0$, so that this eigenvalue ends up with multiplicity 2.
But most importantly, the largest eigenvalue and its eigenvector of this new matrix are just what we were seeking for.
\\
Assuming that $\lVert v_0 \lVert = 1$
\begin{equation}\label{shiftedlaplacian}
	\begin{split}
		\hat{\mathcal{L}} & := \underbrace{2I - \mathcal{L}}_{\text{reverse spectrum}} - \underbrace{2 v_0 v_0^T}_{\text{eigenvalue $2 \rightarrow 0$}}\\
		& = 2 ( I - v_0 v_0^T ) - \mathcal{L}
	\end{split}
\end{equation}

\section{Bringing it all together}
In \eqref{shiftedlaplacian}, we were presented with a simple analytic solution for any spectrally shifted normalized laplacian.
Combining those results with those from (REF MISSING) to fill in the rest:
\begin{equation}
	\begin{split}
		\hat{\mathcal{L}} & = 2 ( I - v_0 v_0^T ) - \mathcal{L}\\
		& = I + D^{-\frac{1}{2}} A D^{-\frac{1}{2}} - 2 v_0 v_0^T
	\end{split}
\end{equation}
New let's see how this operates on some vector $v \in \mathbb{R}^n$. This immediately yields some major optimizations
\begin{equation}
	\begin{split}
		\hat{\mathcal{L}}v & = \left( I + D^{-\frac{1}{2}} A D^{-\frac{1}{2}} - 2 v_0 v_0^T\right) v\\
		& = v + D^{-\frac{1}{2}} A \underbrace{D^{-\frac{1}{2}} v}_{=:\hat{v}} - 2 v_0 \underbrace{v_0^T v}_{=:\frac{\mu}{2}}\\
		& = v + D^{-\frac{1}{2}} A \hat{v} - \mu v_0
	\end{split}
\end{equation}
Thus, the shifted laplacian operator can be realized as a standard matrix multiplication on the original matrix with both the
argument and result vectors being slightly transformed. It it also worth noting that since $D$ is strictly diagonal, the multiplications with $D^{-\frac{1}{2}}$
are actually linear operations in the number of rows. Also: The operator is sparsity-invariant with respect to $A$


\section{The case against other eigensolvers}

%----------------------------------------------------------------------------------------
%	BIBLIOGRAPHY
%----------------------------------------------------------------------------------------

\printbibliography[title={Bibliography}] % Print the bibliography, section title in curly brackets

%----------------------------------------------------------------------------------------

\end{document}
